%--------------------------------
% RESUME IN LaTeX
% Author: Nicholas Harras
% Framework based on Ben Nissan's (https://www.bennissan.github.io/),
% and Sourabh Bajaj's, who is a very famous and talented programmer.
%--------------------------------

% PACKAGES
\documentclass[12pt]{article}
\usepackage{fullpage}
\usepackage{amsmath}
\usepackage{amssymb}
\usepackage[usenames,dvipsnames]{color}
\usepackage{hyperref}
\usepackage[normalem]{ulem}
\usepackage{lipsum}
\usepackage{changepage}

% ADJUST MARGINS
\setlength{\voffset}{-0.2in}
\setlength{\hoffset}{-0.4in}
\setlength{\textheight}{710pt}
\setlength{\textwidth}{540pt}
\pagestyle{empty}
\raggedright

% CUSTOM COMMANDS

% Helper functions
\newcommand{\area}[2]{
	\vspace*{-9pt} 
	\begin{verse}\textbf{#1}   #2 \end{verse}
  	\vspace*{-8pt}
}
\newcommand{\fillpullright}[1]{
	\hfill 
	\hfill 
	\textcolor{Gray}{(#1)}
}
\newcommand{\mailto}[1]{
	\href{mailto:#1}{#1}
}
\newcommand{\reference}[4]{
  	#1 & #2, #3 & \textit{\mailto{#4}}\\
}

% Page Heading (Name/Contact Info)
\newcommand{\contact}[2]{
	\vspace*{-8pt}
	\begin{center}
		{#1}\\ % deleted "\schape{#1}", replaced \LARGE with \huge
		#2
	\end{center}
	\vspace*{-12pt}
}

% Headers
\newcommand{\header}[1]{
	\vspace*{12pt} % increase to create more whitespace
	{\hspace*{-14pt}\vspace*{6pt} #1}
	\vspace*{-6pt} 
	\lineunder
}

% Underline under each category  	
\newcommand{\lineunder}{
	\vspace*{-8pt} \\ 
	\hspace*{-18pt} 
	\hrulefill \\
}

% School
\newcommand{\school}[3]{
 	\textbf{#1} #2 \fillpullright{#3}\\
  	\vspace*{8pt}
}

% Coursework info
\newcommand{\schoolwithcourses}[5]{
  	\textbf{#1} #2 $\bullet$ #3 #4 \fillpullright{#5} \\ 
	\vspace*{8pt}
}

% Employer
\newcommand{\employer}[4]{
	{ \textbf{#1} #2 \fillpullright{#4}\\
    \textit{\fontsize{10}{12}\selectfont #3}\\  }
}

% Subheadings
\newcommand{\subheading}[4]{
 	\vspace{5pt}
    	\begin{tabular*}{1.01\textwidth}
    		{l@{\extracolsep{\fill}}r}
      		\hspace{-16pt}\textbf{#1} & #2 \\
      		\hspace{-16pt}\textit{\small#3} & \textit{\small #4} \\
    	\end{tabular*}
    \vspace{-4pt}
}

% Bodies after subheadings
\newenvironment{achievements}{
\begin{adjustwidth}{-10pt}{}
  \begin{list}{$\bullet$}{
  	\topsep 0pt \itemsep -4pt}}
  	{\vspace*{2pt}\end{list}
\end{adjustwidth}
}

% Custom "C++" look
\def\cpp{
	{\hspace{-0.25em}C\nolinebreak[4]\hspace{-.05em}\raisebox{.4ex}{\tiny		\bf ++}}}
	\def\lildot{$\cdot$
}


\renewcommand{\familydefault}{\sfdefault} % comment out to add serifs


%------START OF DOCUMENT---------
\begin{document}

\small
\smallskip
\vspace*{-40pt}

%------HEADER--------------------
\contact{\huge{N}\LARGE{ICHOLAS} \huge{H}\LARGE{ARRAS}}{201-317-2212 $\bullet$ \mailto{nickharras1@gmail.com} $\bullet$ \href{https://www.github.com/harras}{linkedin.com/in/nicholas-harras/}}


%-----EDUCATION------------------
\vspace{-8pt}
\header{EDUCATION}

\subheading
	{Rutgers University}{New Brunswick, NJ}
	{B.S. in Computer Science}{September 2014 - August 2018}
	\begin{achievements}	
	\item{\bf Related Coursework:} Databases, Internet Technology, Systems Programming, Algorithms, Principles of Programming Languages, Computer Security, Intro to A.I., Computer Architecture, Linear Optimization,   Linear Algebra, Discrete Structures I/II, Calculus I/II, Data Structures
	\end{achievements}

%-----SKILLS AND CREDENTIALS------
\vspace{-8pt}
\header{SKILLS}
	\begin{achievements}
	% 	\item{\bf :}
		\item{\bf Operating Systems:} Windows 10, Ubuntu 16+, RHEL/CentOS 7+, OpenSUSE, Kali Linux 18+, ParrotOS Security 4.11, VirtualBox 6.1, Xen
		\item{\bf Programming and Scripting:} Python 2.7/3, C, Java, Go 1.17, bash, PowerShell
		\item{\bf Databases and Web Dev:} Apache 2.4, PHP 7/8, CodeIgniter, MySQL, PostreSQL
		\item{\bf Security Tools:} Nmap 7.8, WireShark 3+, tcpdump, curl, wget, Metasploit Framework 6, Burp Suite Community Edition, DirBuster, Active Directory, LDAP
		\item{\bf Clearance:} Public Trust
		\item{\bf Certifications:} CompTIA Security+
	\end{achievements}

%-----EXPERIENCE------------------
\vspace{-8pt}
\header{EXPERIENCE}

\subheading
	{Institute for Genomics and Evolutionary Medicine -- Temple University}{Philadelphia, PA}
	{IT Support Sepcialist}{September 2020 - present}
	\begin{achievements}
		
	\end{achievements}

\subheading
	{National Oceanic and Atmospheric Administration}{Princeton, NJ}
	{Computer Operator}{June 2019 - February 2020}
	\begin{achievements}
		\item Utilized \textbf{Slurm} and other command line tools to maintain various Federal \textbf{OpenSUSE Linux} HPC systems that scientists at NOAA's Geophysical Fluid Dynamics Lab rely on for their weather modeling
		\item Responded to or handed off approximately 10 tickets a day within \textbf{OTRS}, in a workplace of about 200 users.	
		\item Answered phones and assisted climatologists and engineers with technical issues, often with \textbf{SSH} tunnelling, \textbf{X2Go} connections, \textbf{Slurm} errors, and compiler issues
		\item Developed \textbf{Python} and \textbf{tcsh} scripts to automate various maintenance tasks
		
		
	\end{achievements}


%-----PROJECTS--------------------
\vspace{-8pt}
\header{PROJECTS}

%\subheading{}{}{}{}
%	\vspace{-15pt}
%	\begin{adjustwidth}{-10pt}{}
%	LOREM IPSUM
%	\end{adjustwidth}
%	\begin{achievements}
%		\item List
%		\item Items
%		\item Here
%	\end{achievements} 
	
\subheading{HPC Charge Code Calculator}{}{}{}
	\vspace{-15pt}
	\begin{adjustwidth}{-10pt}{}
	\textbf{Python} command line tool written to expedite the process of issuing charge codes to HPC systems according to downtime.
	\end{adjustwidth}
	\begin{achievements}
		\item Algorithmically designed to handle any and all possible charge code combinations
		\item Addressed a real workflow bottleneck, reduced time spent calculating charge codes to near-instantaneous
		\item Began as a personal project, but was adopted by the other operators, and became part of our \textbf{GitLab} as an ongoing, official project
	\end{achievements}

\subheading{HPC Monitoring Dashboard}{}{}{}
	\vspace{-15pt}
	\begin{adjustwidth}{-10pt}{}
	\textbf{Grafana} metrics dashboard for various statistics pertinent to monitoring GFDL/NOAA HPCs, utilizing an \textbf{InfluxDB} database and \textbf{Python}
	\end{adjustwidth}
	\begin{achievements}
		\item Developed \textbf{Python} wrappers of \textbf{Slurm} and system functions to constantly update an \textbf{InfluxDB} database
		\item Worked with operators to ensure the user experience of the \textbf{Grafana} displays were helpful to their workflow
		\item Provided reliable uptime logs over time, which proved incredibly helpful for HPC monitoring and troubleshooting, for instance in the event of an outage.
	\end{achievements}

\subheading{Operators' Log Migration}{}{}{}
	\vspace{-15pt}
	\begin{adjustwidth}{-10pt}{}
	Long term project with the goal of moving 30 years of Operators' log from text files to a relational database, utilizing \textbf{Python}, \textbf{SQLite} for testing, and to ultimately write to a \textbf{MySQL} database
	\end{adjustwidth}
	\begin{achievements}
		\item Tested a wide variety of methods to iterate through and parse 200,000 text files, resulting in a final algorithm with a runtime 40\% faster than the original implementation
		\item Sanitized data and implemented quality controls to ensure no data loss
		\item Carefully tested only portions of the logs before determining that we were ready to migrate the logs in its entirety
	\end{achievements} 


%-----CAMPUS ACTIVITIES-----------
%\header{CAMPUS ACTIVITIES}

%\subheading
%	{Undergraduate Student Alliance of Computer Scientists}{Piscataway, NJ}
%	{Mentor/Member}{September 2016 - June 2018}
%	\begin{achievements}
%		\item Taught a group of four CS students the basics of \textbf{Bash}, \textbf{Git}, as well as data structures and other CS concepts
%		\item Offered guidance to CS students with respect to clubs to join, topics to explore, textbooks to read, etc.
%		\item Assisted computer science first-years with their coursework
%	\end{achievements}
	
%\subheading
%	{HackRU}{Piscataway, NJ}
%	{Volunteer}{May 2017}
%	\begin{achievements}
%		\item Assisted competitors throughout the event as a technical advisor, specifically regarding \textbf{Python} and \textbf{Java}
%	\end{achievements}


\end{document}